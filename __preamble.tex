\usepackage[english]{babel}
\usepackage[latin5]{inputenc}

\usepackage{caption}
\usepackage{listings}
\lstset{
	literate={~} {$\sim$}{1}, % set tilde as a literal (no process)      % the language of the code
	basicstyle=\ttfamily\fontsize{7}{8}\selectfont\bfseries,
	frame=tb,
	linewidth=0.98\linewidth,
	columns=flexible,
	numbers=left,                   % where to put the line-numbers
	stepnumber=1,                   % the step between two line-numbers. If it's 1, each line will be numbered
	numbersep=5pt,                  % how far the line-numbers are from the code
	showspaces=false,               % show spaces adding particular underscores
	showstringspaces=false,         % underline spaces within strings
	showtabs=false,                 % show tabs within strings adding particular underscores
	frame=single,                   % adds a frame around the code
	tabsize=2,                      % sets default tabsize to 2 spaces
	captionpos=b,                   % sets the caption-position to bottom
	breaklines=true,                % sets automatic line breaking
	breakatwhitespace=false,        % sets if automatic breaks should only happen at whitespace
	caption=\lstname,                 % show the filename of files included with \lstinputlisting; also try caption instead of title
	escapeinside={\%*}{*)},            % if you want to add a comment within your code
	morekeywords={*,...},               % if you want to add more keywords to the set
	lineskip={-1.2pt},
}

\usepackage{latexsym}
\usepackage{amsfonts}
\usepackage{amssymb}
\usepackage{fancyhdr}
\usepackage{setspace}
\usepackage{multicol}

\usepackage{amsmath}

\usepackage[linesnumbered,ruled]{algorithm2e}
\DontPrintSemicolon
\SetKwInOut{Input}{input}
\SetKwInOut{Output}{output}
\usepackage{setspace}
\usepackage{etoolbox}
\AtBeginEnvironment{algorithm}{\setstretch{1.15}}
\SetKwInOut{Input}{Input}
\SetKwInOut{Output}{Output}

\usepackage{float}
\floatstyle{ruled}
\usepackage{graphicx}
\graphicspath{{figures/}}
\newfloat{myfigure}{thp}{lop}
\floatname{myfigure}{Figure}
\newfloat{mytable}{thp}{lop}
\floatname{mytable}{Table}
\usepackage{multicol}
\usepackage{mathtools}
\DeclarePairedDelimiter\ceil{\lceil}{\rceil}
\DeclarePairedDelimiter\floor{\lfloor}{\rfloor} %usage \floor*{2/3}

\usepackage{enumerate}
\usepackage{latexsym}
\usepackage{graphicx}
\usepackage{amsfonts}
\usepackage{amssymb}
\usepackage{fancyhdr}
\usepackage{setspace}
\usepackage{float}
\usepackage{multicol}
\usepackage{color,soul}
\usepackage{url}
\usepackage{tocloft}
\usepackage{acro}

\usepackage{hyperref}
\hypersetup{%
    pdfborder = {0 0 0},
    colorlinks,
    citecolor=,
    filecolor=,
    linkcolor=,
    urlcolor=
} 
\urlstyle{same}


\usepackage{amsthm}
\newtheorem{theorem}{Theorem}[section]
\newtheorem{corollary}{Corollary}[theorem]
\newtheorem{lemma}[theorem]{Lemma}
\newtheorem{remark}[theorem]{Remark}

\linespread{1.5}
\usepackage{helvet}
\renewcommand{\familydefault}{\sfdefault}
\renewcommand{\rmdefault}{phv} % Arial
\renewcommand{\sfdefault}{phv} % Arial

%Remove chapter numbers from sections.
%\renewcommand*\thesection{\arabic{section}}
\renewcommand*\contentsname{\large{TABLE OF CONTENTS}}
\renewcommand*\listtablename{\large{LIST OF TABLES}}
\renewcommand*\listfigurename{\large{LIST OF FIGURES}}
\renewcommand*\listalgorithmcfname{\large{LIST OF ALGORITHMS}}
\renewcommand*\algorithmcfname{Algorithm}
\renewcommand*\algorithmautorefname{algorithm}
\renewcommand*\lstlistingname{Code}
\renewcommand*\lstlistlistingname{LIST OF CODES}
\DeclareCaptionType{code}[Code Listing][\large{LIST OF CODES}] 

\addto\captionsenglish{\renewcommand{\figurename}{Figure}}
\addto\captionsenglish{\renewcommand{\tablename}{Table}}
\addto\captionsenglish{\renewcommand{\bibname}{References}}
\tolerance=10000
\hyphenpenalty=5000
