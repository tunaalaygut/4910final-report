\university{YASAR UNIVERSTIY}
\faculty{FACULTY OF ENGINEERING}
\dept{DEPARTMENT OF COMPUTER ENGINEERING}
\course{COMP4910 Senior Design Project 1, Fall 2019}
\advisor{\large{Supervisor}: {Mutlu BEYAZIT}}
\title{{Project Code Name}: {SLATE}}
\date{{28.12.2019}}
%\submitted{Izmir, 2017} %This field has been disabled Do not use it.

\author{
	{Tuna ALAYGUT}, Student ID: {15070001002}\\
	{Berkay BAYINDIR}, Student ID: {16070001002}\\
	{Alara İŞCAN}, Student ID: {15070001016}\\
}


\revision{Revision\,1.0}
%\revision{Revision\,2.0}
\revision{Final Report}


\dedication{
	\hl{You can dedicate your thesis.}
}

\acknowledgements{
	{The acknowledgements are here.}
}

\keywords{ 
	{convolutional neural network, sign language interpretation, computer vision}
}

\abstract{
	{Slate, is a project designed to facilitate the daily communication of speech/hearing impaired. Fundamentally, it consists of three main components. First component is an external display which is capable of working with a generic smartphone and is located at the back of the smartphone. Second component, in the heart of Slate, is an artificial intelligence which detects, segmentates and classifies hand gestures in the sign language alphabet. Third component is a smart phone application which is resposible for communicating the artificial intelligence and external display components. It also provides an interface to the users of the project.\\Artificial intelligence component does the reading the hand gestures from smartphone camera, classification of the gesture and the application component transfers it to the external display. Making it easier for speech/hearing impaired to engage in daily conversations. Slate, acts as an interpreter in these conversations.}
}

\ozet{
	{Slate, işitme/konuşma engelli, işaret dili kullanan kişilerin günlük hayattaki iletişimlerini kolaylaştırmaya yönelik tasarlanan bir projedir. Temelde üç bileşenden oluşmaktadır. Bu bileşenlerden ilki akıllı telefonlar ile birlikte çalışabilen, telefonun arkasında konumlandırılacak bir dış ekran ünitesidir. İkinci bileşen ise projenin merkezinde bulunan, işaret dili alfabesini tanıyıp, sınıflandırıp, yazıya çeviren bir yapay zeka bileşenidir. Üçüncü ve son bileşen ise ilk iki bileşenin iletişiminden sorumlu olan ve kullanım kolaylığı sağlayan akıllı telefon uygulamasıdır.\\ Yapay zeka bileşini, akıllı telefonun kamerasından aldığı ve yazıya dönüştürdüğü işaret dili karakterlerini dış ekran ünitesine yollar. Böylece işaret dili kullanan kişilerin, karşılarındaki kişiler ile iletişimi sağlanmış olur. Slate bu iletişimde bir tercüman rolü oynar.}
}
